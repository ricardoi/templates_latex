\documentclass{NSF}

\graphicspath{{figures/}}

\begin{document}

% A. Cover Sheet
% A number of the boxes contained on the Cover Sheet are
% electronically pre-filled as part of the FastLane login process
% Complete the rest of your info there

% B. Project Summary
\title{NSF Proposal Template}
\newsection{B}
%\input{sections/summary}
\section{Project Summary}
\subsection{Overview} 
Each proposal must contain a summary of the proposed project not more than {\bf one page in length}. The Project
Summary consists of an overview, a statement on the intellectual merit of the proposed activity, and a statement
on the broader impacts of the proposed activity.
The overview includes a description of the activity that would result if the proposal were funded and a statement
of objectives and methods to be employed.  

The Project Summary should be written in the third person, informative to other persons working in
the same or related fields, and, insofar as possible, understandable to a scientifically or technically 
literate lay reader. It should not be an abstract of the proposal.

If the Project Summary contains special characters it may be uploaded as a Supplementary Document.
{\bf Project Summaries submitted as a PDF must be formatted with separate headings for the overview, statement on the
intellectual merit of the proposed activity, and statement on the broader impacts of the proposed activity}. Failure
to include these headings may result in the proposal being returned without review.
Additional instructions for preparation of the Project Summary are available in FastLane.\\
\subsection{Intellectual Merit} 
The statement on intellectual merit should describe the potential of the proposed activity to advance knowledge.
\subsection{Broader Impacts of the Proposed Work} 
The statement on broader impacts should describe the potential of the proposed activity to benefit society and contribute to the achievement of specific, desired societal outcomes.
% C. Table of Contents 
% A Table of Contents is automatically generated for the proposal by FastLane

% D. Project Description
\newpage\newsection{D}
%\input{sections/description}
\section{Project Description}
\subsection{Introduction}
\subsection{Proposed Study}
The Project Description should provide a clear statement of the work to be undertaken and must include:
objectives for the period of the proposed work and expected significance; relation to longer-term goals of the PI's
project; and relation to the present state of knowledge in the field, to work in progress by the PI under other
support and to work in progress elsewhere.

The Project Description should outline the general plan of work, including the broad design of activities to be
undertaken, and, where appropriate, provide a clear description of experimental methods and procedures.
Proposers should address what they want to do, why they want to do it, how they plan to do it, how they will
know if they succeed, and what benefits could accrue if the project is successful. The project activities may be
based on previously established and/or innovative methods and approaches, but in either case must be well
justified. These issues apply to both the technical aspects of the proposal and the way in which the project may
make broader contributions.

\subsection{Broader Impacts of the Proposed Work}
The Project Description must contain, as a separate section within the narrative, a section labeled ``Broader
Impacts of the Proposed Work". This section should provide a discussion of the broader impacts of the proposed
activities. Broader impacts may be accomplished through the research itself, through the activities that are
directly related to specific research projects, or through activities that are supported by, but are complementary to 
the project. NSF values the advancement of scientific knowledge and activities that contribute to the
achievement of societally relevant outcomes. Such outcomes include, but are not limited to: full
participation of women, persons with disabilities, and underrepresented minorities in science, technology, engineering, and
mathematics (STEM); improved STEM education and educator development at any level; increased public
scientific literacy and public engagement with science and technology; improved well-being of individuals in
society; development of a diverse,globally competitive STEM workforce; increased partnerships between
academia, industry, and others; improved national security; increased economic competitiveness of the United
States; and enhanced infrastructure for research and education.

\subsection{Results from Prior NSF Support}
If any PI or co-PI identified on the project has received NSF funding (including any current
funding) in the past five years, in formation on the award(s) is required,
irrespective of whether the support was directly related to the proposal or not.
In cases where the PI or co-PI has received more than one award (excluding amendments),
they need only report on the one award most closely related to the proposal. Funding includes not just salary
support, but any funding awarded by NSF. The following information must be provided:\\

\noindent
\emph{\underline{Name of PI}}: NSF-Program (Award Number) ``Title of the Project'' (\$AMOUNT, PERIOD OF SUPPORT). 
{\bf Publications:} List of publications resulting from the NSF award. A complete bibliographic citation for each
publication must be provided either in this section or in the References Cited section of the proposal); if
none, state: ``No publications were produced under this award.'' {\bf Research Products:} evidence of research products 
and their availability, including, but not limited to: data, publications, samples, physical collections, software, 
and models, as described in any Data Management Plan.

% E. References Cited
\newpage\newsection{E}
\renewcommand\refname{References Cited}
\bibliography{references}
% I prefer to use the IEEE bibliography style. 
% That's  NOT required by the NSF guidelines. 
% Feel Free to use whatever style you prefer
\bibliographystyle{IEEEtran}

% F. Biographical Sketch(es)
\newpage\newsection{F}
%\input{sections/bio}

% G. Budget Justification
\newpage\newsection{G}
%\input{sections/budget}
\section{Budget Justification}
% No more than 3 pages!!! 
\subsection{A. Senior Personnel}
\noindent{\bf A1.} Includes PI at 10\% CY.
\subsection{B. Other Personnel}
\noindent{\bf B3.} Includes stipend for one graduate student for each calendar year of the project.  
\subsection{C. Fringe Benefits}
Fringe benefits are calculated at a rate of X\% for faculty, Y\% for graduate students.  
\subsection{E. Travel}
1) all travel (both domestic and foreign) must now be justified. 
2) temporary dependent care costs above and beyond regular dependent care that directly result from travel to conferences are allowable costs provided that the conditions established in 2 CFR � 200.474 are met.
\subsection{G. Other Direct Costs}
1) Includes coverage on costs of computing devices
2) The charging of computing devices as a direct cost is allowable for devices that are essential and allocable, but not solely dedicated, to the performance of the NSF award
\noindent{\bf G5.} Includes tuition for graduate students participating in the program.
\subsection{H. Indirect Costs}
Overhead at a rate of X\% is charged on all direct salaries and wages, applicable fringe benefits, materials and supplies, services, travel and subawards up to the first \$X of each subaward. Excluded are equipment and the portion of each subaward in excess of \$X.


%  H. Current and Pending Support
\newpage\newsection{H}
%\input{sections/support}
\section{Current \& Pending Support}
\begin{tabular}{ll}
\textbf{Investigator:} 			& \\
\textbf{Project Title:}			& Put your Proposal title here\\
\textbf{Project Location:}		& \\
\textbf{Source of Support:} 	& NSF\\
\textbf{Total Award Amount:} 	& \\
\textbf{Total Award Period:}	& \\
\textbf{Status:}				& Pending (this project) \\
\end{tabular}

% I. Facilities, Equipment and Other Resources
\newpage\newsection{I}
%\input{sections/resources}
\section{Facilities, Equipments, \& other Resources}
This section of the proposal is used to assess the adequacy of the resources available to perform the effort
proposed to satisfy both the Intellectual Merit and Broader Impacts review criteria. Proposers should describe
only those resources that are directly applicable. Proposers should include an aggregated description of the
internal and external resources (both physical and personnel) that the organization and its collaborators will
provide to the project, should it be funded. Such information must be provided in this section, in lieu of other
parts of the proposal (e.g., budget justification, project description). The description should be narrative in nature
and must not include any quantifiable financial information. Reviewers will evaluate the information during the
merit review process and the cognizant NSF Program Officer will review it for programmatic and technical
sufficiency.

% J. Special Information and Supplementary Documentation
\newpage\newsection{J}
%\input{sections/data}		% Data Management Plan (Required)
%\input{sections/postdoc} % Postdoctoral Researcher Mentoring Plan (if applicable)
\section{Data Management Plan}
Proposals must include a supplementary document of no more than two pages labeled ``Data Management Plan". This
supplementary document should describe how the proposal will conform to NSF policy on the
dissemination and sharing of research results (see AAG Chapter VI.D.4)

\end{document}
