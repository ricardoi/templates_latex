\documentclass{article}
\usepackage{fullpage}
\usepackage{parskip}
\usepackage{titlesec}
\usepackage{xcolor}
\usepackage[colorlinks = true,
            linkcolor = blue,
            urlcolor  = blue,
            citecolor = blue,
            anchorcolor = blue]{hyperref}
\usepackage[natbibapa]{apacite}
\usepackage{eso-pic}
\AddToShipoutPictureBG{\AtPageLowerLeft{\includegraphics[scale=0.7]{powered-by-Authorea-watermark.png}}}

\renewenvironment{abstract}
  {{\bfseries\noindent{\abstractname}\par\nobreak}\footnotesize}
  {\bigskip}

\titlespacing{\section}{0pt}{*3}{*1}
\titlespacing{\subsection}{0pt}{*2}{*0.5}
\titlespacing{\subsubsection}{0pt}{*1.5}{0pt}

\usepackage{authblk}

\usepackage{graphicx}
\usepackage[space]{grffile}
\usepackage{latexsym}
\usepackage{textcomp}
\usepackage{longtable}
\usepackage{tabulary}
\usepackage{booktabs,array,multirow}
\usepackage{amsfonts,amsmath,amssymb}
\providecommand\citet{\cite}
\providecommand\citep{\cite}
\providecommand\citealt{\cite}
% You can conditionalize code for latexml or normal latex using this.
\newif\iflatexml\latexmlfalse
\providecommand{\tightlist}{\setlength{\itemsep}{0pt}\setlength{\parskip}{0pt}}%

\AtBeginDocument{\DeclareGraphicsExtensions{.pdf,.PDF,.eps,.EPS,.png,.PNG,.tif,.TIF,.jpg,.JPG,.jpeg,.JPEG}}

\usepackage[utf8]{inputenc}
\usepackage[english]{babel}



\begin{document}

\title{The ISME Journal Template\\}


 \author{Linus Pauling}


\date{}

\begingroup
\let\center\flushleft
\let\endcenter\endflushleft
\maketitle
\endgroup




\subsection*{Preparation of Articles}\label{auto-label-subsection-804801}

Please note that original articles must contain the following
components.\\

Please see below for further details.\\

\begin{itemize}
\tightlist
\item
  Cover letter\\
\item
  Title page (excluding acknowledgements)\\
\item
  Abstract\\
\item
  Introduction\\
\item
  Materials (or Subjects) and Methods\\
\item
  Results\\
\item
  Discussion\\
\item
  Acknowledgements\\
\item
  Conflict of Interest\\
\item
  References\\
\item
  Figure legends\\
\item
  Tables\\
\item
  Figures\\
\end{itemize}

Reports of clinical trials must adhere to the registration and reporting
requirements listed in the Editorial Policies.\\

\textbf{Cover Letter:}\\

The uploaded covering letter must state the material is original
research, has not been previously published and has not been submitted
for publication elsewhere while under consideration. If the manuscript
has been previously considered for publication in another journal,
please include the previous reviewer comments, to help expedite the
decision by the Editorial team. Please include a Conflict of Interest
statement.\\

\textbf{Title Page:}\\

The title page should bear the title of the paper, the full names of all
the authors and their affiliations, together with the name, full postal
address, telephone and fax numbers and e-mail address of the author to
whom correspondence and offprint requests are to be sent (this
information is also asked for on the electronic submission form). The
title page must also contain a Conflict of Interest statement (see
Editorial Policy section).\\

\begin{itemize}
\tightlist
\item
  The title should be brief, informative, of 150 characters or less and
  should not make a statement or conclusion.\\
\item
  The running title should consist of no more than 50 letters and
  spaces. It should be as brief as possible, convey the essential
  message of the paper and contain no abbreviations.\\
\item
  Authors should disclose the sources of any support for the work,
  received in the form of grants and/or equipment and drugs.\\
\item
  If authors regard it as essential to indicate that two or more
  co-authors are equal in status, they may be identified by an asterisk
  symbol with the caption `These authors contributed equally to this
  work' immediately under the address list.\\
\end{itemize}

\textbf{Subject Categories:}\\

The Subject Categories are used to structure the current and archived
online content of The ISME Journal, and to help readers interested in
particular areas of microbial ecology find relevant information more
easily. Subject Categories are also indicated in the table of contents
and on the title page of the published article. Authors should suggest
an appropriate Subject Category for the submitted manuscript. One
category may be selected from the following list:\\

\begin{itemize}
\tightlist
\item
  Microbial population and community ecology\\
\item
  Microbe-microbe and microbe-host interactions\\
\item
  Evolutionary genetics\\
\item
  Integrated genomics and post-genomics approaches in microbial
  ecology\\
\item
  Microbial engineering\\
\item
  Geomicrobiology and microbial contributions to geochemical cycles\\
\item
  Microbial ecology and functional diversity of natural habitats\\
\item
  Microbial ecosystem impacts\\
\end{itemize}

\textbf{Abstract:}\\

Original Articles must be prepared with an unstructured abstract
designed to summarise the essential features of the paper in a logical
and concise sequence.\\

\textbf{Materials/Subjects and Methods:}\\

This section should contain sufficient detail, so that all experimental
procedures can be reproduced, and include references. Methods, however,
that have been published in detail elsewhere should not be described in
detail. Authors should provide the name of the manufacturer and their
location for any specifically named medical equipment and instruments,
and all drugs should be identified by their pharmaceutical names, and by
their trade name if relevant.\\

\textbf{Results and Discussion:}\\

The Results section should briefly present the experimental data in
text, tables or figures. Tables and figures should not be described
extensively in the text, either. The discussion should focus on the
interpretation and the significance of the findings with concise
objective comments that describe their relation to other work in the
area. It should not repeat information in the results. The final
paragraph should highlight the main conclusion(s), and provide some
indication of the direction future research should take.\\

\textbf{Acknowledgements:}\\

These should be brief, and should include sources of support including
sponsorship (e.g. university, charity, commercial organisation) and
sources of material (e.g. novel drugs) not available commercially.\\

\textbf{Conflict of Interest:}\\

Authors must declare whether or not there are any competing financial
interests in relation to the work described. This information must be
included at this stage and will be published as part of the paper.
Conflict of interest should be noted in the cover letter and also on the
title page. Please see the Conflict of Interest documentation in the
Editorial Policy section for detailed information.\\

\textbf{References:}\\

Only papers directly related to the article should be cited. Exhaustive
lists should be avoided. References should follow the Havard format. In
the text of the manuscript, a reference should be cited by author and
year of publication eg (Bailey \& Kowalchuk, 2006) and (Heidelberg et
al, 1994) and listed at the end of the paper in alphabetical order of
first author. References should be listed and journal titles abbreviated
according to the style used by Index Medicus, examples are given
below.\\

All authors should be listed for papers with up to six authors; for
papers with more than six authors, the first six only should be listed,
followed by et al. Abbreviations for titles of medical periodicals
should conform to those used in the latest edition of Index Medicus. The
first and last page numbers for each reference should be provided.
Abstracts and letters must be identified as such. Papers in press may be
included in the list of references.\\

Personal communications must be allocated a number and included in the
list of references in the usual way or simply referred to in the text;
the authors may choose which method to use. In either case authors must
obtain permission from the individual concerned to quote his/her
unpublished work.\\

Examples:\\

Journal article: Cho JC, Kim MW, Lee DH, Kim SJ. (1997). Response of
bacterial communities to changes in composition of extracellular organic
carbon from phytoplankton in Daechung reservoir (Korea). Arch Hydrobiol
138:559--576.\\

Journal article, e-pub ahead of print: Eng-Kiat L, Bowles DJ. A class of
plant glycosyltransferases involved in cellular homeostasis. EMBO J
2004; e-pub ahead of print 8 July 2004, doi: 10.1038/sj.emboj.7600295.\\

Journal article, in press: Lim E-K, Ashford DA, Hou B, Jackson RG,
Bowles DJ. (2004). Arabidopsis glycosyltransferases as biocatalysts in
fermentation for regioselective synthesis of diverse quercetin
glucosides. Biotech Bioeng (in press).\\

Complete book: Sambrook J, Fritsch E, Maniatis T. (1989). Molecular
Cloning: a Laboratory Manual. Cold Spring Harbor Press: New York.
Chapter in book: Zinder, SH. (1998). Methanogens. In: Burlage, RS (ed).
Techniques in Microbial Ecology. Oxford University Press: Oxford, pp
113-- 136.~\\

\textbf{Tables:}\\

Tables should only be used to present essential data; they should not
duplicate what is written in the text. It is imperative that any tables
used are editable, ideally presented in Excel. Each must be uploaded as
a separate workbook with a title or caption and be clearly labelled,
sequentially. Please make sure each table is cited within the text and
in the correct order, e.g. (Table 3). Please save the files with
extensions .xls / .xlsx / .ods / or .doc or .docx. Please ensure that
you provide a `flat' file, with single values in each cell with no
macros or links to other workbooks or worksheets and no calculations or
functions.\\

\textbf{Figures:}\\

Figures and images should be labelled sequentially and cited in the
text. Figures should not be embedded within the text but rather uploaded
as separate files. Detailed guidelines for submitting artwork can be
found by downloading our\\

Artwork Guidelines. The use of three-dimensional histograms is strongly
discouraged when the addition of the third dimension gives no extra
information.\\

\textbf{Artwork Guidelines:}\\

Detailed guidelines for submitting artwork can be found by downloading
the guidelines PDF. Using the guidelines, please submit production
quality artwork with your initial online submission. If you have
followed the guidelines, we will not require the artwork to be
resubmitted following the peerreview process, if your paper is accepted
for publication.\\

\textbf{Colour on the web:}\\

Authors who wish their articles to have FREE colour figures on the web
(only available in the HTML (full text) version of manuscripts) must
supply separate files in the following format. These files should be
submitted as supplementary information and authors are asked to mention
they would like colour figures on the web in their submission letter.\\

\textbf{Reuse of Display Items:}\\

See the Editorial Policy section for information on using previously
published tables or figures. Standard abbreviations: Because the
majority of readers will have experience in microbial ecology, the
journal will accept papers which use certain standard abbreviations,
without definition in the summary or in the text. Non-standard
abbreviations should be defined in full at their first usage in the
Summary and again at the first usage in the text, in the conventional
manner. If a term is used 1-4 times in the text, it should be defined in
full throughout the text and not abbreviated.\\

\textbf{Supplementary Information:}\\

Supplementary information (SI) is peer-reviewed material directly
relevant to the conclusion of an article that cannot be included in the
printed version owing to space or format constraints. The article must
be complete and selfexplanatory without the SI, which is posted on the
journal's website and linked to the article. SI may consist of data
files, graphics, movies or extensive tables. Please see our Artwork
Guidelines for information on accepted file types. Authors should submit
supplementary information files in the FINAL format as they are not
edited, typeset or changed, and will appear online exactly as submitted.
When submitting SI, authors are required to:\\

\begin{itemize}
\tightlist
\item
  Include a text summary (no more than 50 words) to describe the
  contents of each file.\\
\item
  Identify the types of files (file formats) submitted.\\
\item
  Include the text ``Supplementary information is available at (journal
  name)'s website'' at the end of the article and before the
  references.\\
\end{itemize}

\section*{Results}\label{results}

This section is only included in papers that rely on primary research.
This section catalogues the results of the experiment. The results
should be presented in a clear and unbiased way. Most results sections
will contain \href{http://authorea.com}{links}~as well as
citations~\hyperref[csl:1]{[1]}~and equations such as~\(e^{i\pi}+1=0\)\\

\section*{Conclusion}\label{auto-label-section-853974}

The conclusion should reinforce the major claims or interpretation in a
way that is not mere summary. The writer should try to indicate the
significance of the major claim/interpretation beyond the scope of the
paper but within the parameters of the field. The writer might also
present complications the study illustrates or suggest further research
the study indicates is necessary.

\selectlanguage{english}
\FloatBarrier
\section*{References}\sloppy
\phantomsection
\label{csl:1}1. Einstein A. {Näherungsweise Integration der Feldgleichungen der Gravitation}. In: \textit{Albert Einstein: Akademie-Vorträge}. 1916. Wiley-Blackwell, pp 99–108. 

\end{document}

